%%%%%%%%%%%%%%%%%%%%%%%%%%%%%%%%%%%%%%%%%%%%%%%%%%%%%%%%%%%%%%%%%%%%%%%%%%%%%%%
%
% apresentacao2.tex
%
% Este é um exemplo de apresentação em latex-beamer usando o tema "Urubu",
% um tema extra-oficial feito para a comunidade do Instituto de Matemática 
% da UFRJ (Universidade Federal do Rio de Janeiro).
%
% Este arquivo é parte do pacote `beamerthemeUrubu`, disponibilizado em
% 
%     https://github.com/~rmsrosa
%
% Mais informações sobre o tema no arquivo README.md, nas apresentações
% disponibilzadas no pacote e nos comentários do arquivo 
% `beamerthemeUrubu.sty`
%
% Written by Ricardo M. S. Rosa <http://www.im.ufrj.br/rrosa>
%
% Copyright (c) 2019 Ricardo M S Rosa <rrosa@im.ufrj.br>
%
% This work may be distributed and/or modified under the
% conditions of the LaTeX Project Public License, either 
% version 1.3 (LPPL-1.3c) of this license or any later version.
%
% The latest version of this license is in
%
%   http://www.latex-project.org/lppl.txt
%
% and version 1.3 or later is part of all distributions of LaTeX
% version 2005/12/01 or later.
%
% This work has the LPPL maintenance status `maintained'.
% 
% The Current Maintainer of this work is Ricardo M. S. Rosa
%
% For the sake of this license, this work consists exclusively of the 
% file `apresentacao2.tex`.
%
%%%%%%%%%%%%%%%%%%%%%%%%%%%%%%%%%%%%%%%%%%%%%%%%%%%%%%%%%%%%%%%%%%%%%%%%%%%%%%%

\documentclass{beamer}

%%%%%%%%%%%%%%%%%%%%%%%%%%%%%%%%%%%%%%%%%%%%%%%%%%%%%%%%%%%%%%%%%%%%%%%%%%%%%%%
% Load Packages
%

% load language related packages:
\usepackage[utf8]{inputenc}
\usepackage[T1]{fontenc}
\usepackage[brazil]{babel}

% load improved tables package
\usepackage{booktabs}

% load Urubu theme with \usetheme{Urubu}
%\usetheme[CemAnos]{Urubu}

% if loading from a directory relative to the current file's path, 
% must use \usepackage{relative/path/to/beamerthemeUrubu} instead:
\usepackage[footline]{../theme/sty/beamerthemeUrubu}

%%%%%%%%%%%%%%%%%%%%%%%%%%%%%%%%%%%%%%%%%%%%%%%%%%%%%%%%%%%%%%%%%%%%%%%%%%%%%%%
% Presentation commands and macros
%
% New command to highlight text with color from the example block
\newcommand{\highlight}[1]{{\usebeamercolor[fg]{block title example}#1}}

%%%%%%%%%%%%%%%%%%%%%%%%%%%%%%%%%%%%%%%%%%%%%%%%%%%%%%%%%%%%%%%%%%%%%%%%%%%%%%%
% Define titlepage data
% 
\title[Tema Urubu para o \LaTeX-beamer]{Apresentação 2: O tema \emph{Urubu} para o \LaTeX-beamer com outras opções}
\author[Ricardo M S Rosa]{Ricardo M. S. Rosa}
\institute[IM-UFRJ]{Instituto de Matemática  \\ Universidade Federal do Rio de Janeiro}
\info{Disponível em \href{https://github.com/rmsrosa/}{github.com/rmsrosa}}
\date[26/Nov/2019]{26 de novembro de 2019}

\begin{document}

\begin{frame}
  \titlepage
\end{frame}

\begin{frame}[fragile]{Opções do tema}
  
  Este tema possui as seguintes opções:

  \begin{enumerate}
    \item \highlight{Logomarca da UFRJ:}
      \begin{enumerate}[a.]
        \item Usar a logomarca Minerva da UFRJ, que é a opção \emph{default}, ou 
        \item Usar a logomarca de Cem Anos da UFRJ, com a opção \highlight{\texttt{CemAnos}};
      \end{enumerate}
    \item \highlight{Posição da logomarca do IM:}
      \begin{enumerate}[a.]
        \item Usar a logomarca do IM no topo da página, que é a opção \emph{default}, ou 
        \item usar a logomarca do IM no pé de página, com a opção \highlight{\texttt{footline}};
      \end{enumerate}
  \end{enumerate}

  Desta vez, estamos usando a opção \highlight{\texttt{footline}} para a posição da logomarca do IM e deixando a opção \emph{default} de usar a logomarca Minerva da UFRJ. Isso é obtido com um dos seguintes comandos no preâmbulo:
  \begin{center}
    {\small\verb|\usetheme[footline]{Urubu}|}
  \end{center}
  ou 
  \begin{center}
    {\small\verb|\usepackage[footline]{path/to/theme/sty/beamerthemeUrubu}|}
  \end{center}

\end{frame}

\begin{frame}{1\-2\-3\-4\-5\-6\-7\-8\-9\-1\-1\-2\-3\-4\-5\-6\-7\-8\-9\-2\-1\-2\-3\-4\-5\-6\-7\-8\-9\-3\-1\-2\-3\-4\-5\-6\-7\-8\-9\-4\-1\-2\-3\-4\-5\-6\-7\-8\-9\-5}

  Veja que, com a logomarca do IM no pé de página, cada linha do título do \emph{frame} abriga até 50 caracteres. 
  \medskip
  
  Como pode ser visto na \href{run:apresentacao1.pdf}{Apresentação 1}, com a logomarca em cima, cada linha do título do \emph{frame} abriga, no máximo, 42 caracteres, ou seja, 16\% a menos.

\end{frame}

\begin{frame}

  \centerline{\Large Fim!}

\end{frame}

\end{document}