%%%%%%%%%%%%%%%%%%%%%%%%%%%%%%%%%%%%%%%%%%%%%%%%%%%%%%%%%%%%%%%%%%%%%%%%%%%%%%%
%
% apresentacao1.tex
%
% Este é um exemplo de apresentação em latex-beamer usando o tema "Urubu",
% um tema extra-oficial feito para a comunidade do Instituto de Matemática 
% da UFRJ (Universidade Federal do Rio de Janeiro).
%
% Este arquivo é parte do pacote `beamerthemeUrubu`, disponibilizado em
% 
%     https://github.com/~rmsrosa
%
% Mais informações sobre o tema no arquivo README.md, nas apresentações
% disponibilzadas no pacote e nos comentários do arquivo 
% `beamerthemeUrubu.sty`
%
% Written by Ricardo M. S. Rosa <http://www.im.ufrj.br/rrosa>
%
% Copyright (c) 2019 Ricardo M S Rosa <rrosa@im.ufrj.br>
%
% This work may be distributed and/or modified under the
% conditions of the LaTeX Project Public License, either 
% version 1.3 (LPPL-1.3c) of this license or any later version.
%
% The latest version of this license is in
%
%   http://www.latex-project.org/lppl.txt
%
% and version 1.3 or later is part of all distributions of LaTeX
% version 2005/12/01 or later.
%
% This work has the LPPL maintenance status `maintained'.
% 
% The Current Maintainer of this work is Ricardo M. S. Rosa
%
% For the sake of this license, this work consists exclusively of the 
% file `apresentacao1.tex`.
%
%%%%%%%%%%%%%%%%%%%%%%%%%%%%%%%%%%%%%%%%%%%%%%%%%%%%%%%%%%%%%%%%%%%%%%%%%%%%%%%

\documentclass{beamer}

%%%%%%%%%%%%%%%%%%%%%%%%%%%%%%%%%%%%%%%%%%%%%%%%%%%%%%%%%%%%%%%%%%%%%%%%%%%%%%%
% Load Packages
%

% load language related packages:
\usepackage[utf8]{inputenc}
\usepackage[T1]{fontenc}
\usepackage[brazil]{babel}
%
% load improved tables package
\usepackage{booktabs}

% load Urubu theme with \usetheme{Urubu}
%\usetheme[CemAnos]{Urubu}

% if loading from a directory relative to the current file's path, 
% must use \usepackage{relative/path/to/beamerthemeUrubu} instead
\usepackage[CemAnos]{../theme/sty/beamerthemeUrubu}

%%%%%%%%%%%%%%%%%%%%%%%%%%%%%%%%%%%%%%%%%%%%%%%%%%%%%%%%%%%%%%%%%%%%%%%%%%%%%%%
% Presentation commands and macros
%
% New command to highlight text with color from the example block
\newcommand{\highlight}[1]{{\usebeamercolor[fg]{block title example}#1}}

%%%%%%%%%%%%%%%%%%%%%%%%%%%%%%%%%%%%%%%%%%%%%%%%%%%%%%%%%%%%%%%%%%%%%%%%%%%%%%%
% The following code extracts the main colors used in the theme
% 
\newcommand{\defcolorfrombeamer}[1]{{\usebeamercolor{#1}\definecolor{#1.fg}{named}{fg}}{\usebeamercolor{#1}\definecolor{#1.bg}{named}{bg}}}

\defcolorfrombeamer{frame title}
\defcolorfrombeamer{block title}
\defcolorfrombeamer{block body}
\defcolorfrombeamer{block title alerted}
\defcolorfrombeamer{block body alerted}
\defcolorfrombeamer{block title example}
\defcolorfrombeamer{block body example}

\newcommand{\extractcolorcode}[2]{\extractcolorspecs{#2}{\auxmodel}{\auxcolor} \convertcolorspec{\auxmodel}{\auxcolor}{#1}\printcol \printcol}

\newcommand{\linhadecor}[1]{{\color{#1}#1} & {\color{#1}$\blacksquare$} & \extractcolorcode{RGB}{#1} & \extractcolorcode{HTML}{#1}}

%%%%%%%%%%%%%%%%%%%%%%%%%%%%%%%%%%%%%%%%%%%%%%%%%%%%%%%%%%%%%%%%%%%%%%%%%%%%%%%
% Define titlepage data
% 
\title[Tema Urubu para o \LaTeX-beamer]{Apresentação 1: O tema \emph{Urubu} para o \LaTeX-beamer}
\subtitle{Um tema para a comunidade do Instituto de Matemática da UFRJ}
\author[Ricardo M S Rosa]{Ricardo M. S. Rosa}
\institute[IM-UFRJ]{Instituto de Matemática  \\ Universidade Federal do Rio de Janeiro}
\info{Disponível em \href{https://github.com/rmsrosa/}{github.com/rmsrosa}}
\date[26/Nov/2019]{26 de novembro de 2019}

%%%%%%%%%%%%%%%%%%%%%%%%%%%%%%%%%%%%%%%%%%%%%%%%%%%%%%%%%%%%%%%%%%%%%%%%%%%%%%%
% Start presentation contents
% 
\begin{document}

\begin{frame}
  \titlepage
\end{frame}

\begin{frame}[fragile]{Title page options}

  O comando {\small\verb|\titlepage|} monta a página com todos os campos usuais:
  \begin{itemize}
    \item {\small\verb|\title[short title]{long title}|}
    \item {\small\verb|\subtitle{subtitle}|}
    \item {\small\verb|\author[short name]{long name}|}
    \item {\small\verb|\institute[short name]{long name}|}
    \item {\small\verb|\date[short date]{long date}|}
    \item {\small\verb|\addtobeamertemplate{title page}{text before}{text after}|}
    \item {\small\verb|\titlegraphic{include graphics with the package you like}|}
  \end{itemize}
  Além disso, como gosto de incluir informações sobre o evento entre o instituto e a data, incluí, entre os dois, um campo para informações extras:
  \begin{itemize}
    \item {\small\verb|\info{meeting info}|}
  \end{itemize}

\end{frame}

\begin{frame}[fragile]{Opções do tema}

  Este tema possui as seguintes opções:

  \begin{enumerate}
    \item \highlight{Logomarca da UFRJ:}
      \begin{enumerate}[a.]
        \item Usar a logomarca Minerva da UFRJ, que é a opção \emph{default}, ou 
        \item Usar a logomarca de Cem Anos da UFRJ, com a opção \highlight{\texttt{CemAnos}};
      \end{enumerate}
    \item \highlight{Posição da logomarca do IM:}
      \begin{enumerate}[a.]
        \item Usar a logomarca do IM no topo da página, que é a opção \emph{default}, ou 
        \item usar a logomarca do IM no pé de página, com a opção \highlight{\texttt{footline}};
      \end{enumerate}
  \end{enumerate}

  Nesta apresentação, estamos usando a opção \highlight{\texttt{CemAnos}} para a logomarca da UFRJ e deixando a posição da logomarca do IM na opção \emph{default}. Isso é obtido com um dos seguintes comandos no preâmbulo:
  \begin{center}
    {\small\verb|\usetheme[CemAnos]{Urubu}|}
  \end{center}
  ou 
  \begin{center}
    {\small\verb|\usepackage[CemAnos]{path/to/theme/sty/beamerthemeUrubu}|}
  \end{center}

\end{frame}


\begin{frame}{Cores do tema}

  \begin{center}
    \begin{tabular}{cccc}
      \hline
      cor & bloco & RGB & HTML \\
      \hline
      \linhadecor{frame title.fg} \\
      \linhadecor{file color} \\
      \linhadecor{url color} \\
      \linhadecor{block title.fg} \\
      \linhadecor{block title.bg} \\
      \linhadecor{block body.fg} \\
      \linhadecor{block body.bg} \\
      \linhadecor{block title alerted.fg} \\
      \linhadecor{block title alerted.bg} \\
      \linhadecor{block body alerted.bg} \\
      \linhadecor{block title example.fg} \\
      \linhadecor{block title example.bg} \\
      \linhadecor{block body example.bg} \\
      \hline
    \end{tabular}
  \end{center}

\end{frame}

\begin{frame}{Cores dos blocos em ação}

  \begin{block}{Teorema (Fórmula de Euler)}
    Para cada $\theta\in \mathbb{R}$, segue que $e^{i\theta} = \cos\theta + i\sin\theta.$
  \end{block}
  
  \begin{exampleblock}{Exemplo (Identidade de Euler)}
    Quando $\theta = \pi$, temos o caso especial $e^{i\pi} + 1 = 0.$
  \end{exampleblock}
  
  \begin{alertblock}{Observação}
    A \textbf{Fórmula de Euler} se estende para um complexo qualquer $\alpha + i\beta$, $\alpha, \beta\in \mathbb{R}$, nos dando 
    $ e^{\alpha + i\beta} = e^{\alpha}(\cos\beta + i\sin\beta).
    $
  \end{alertblock}
\end{frame}

\begin{frame}[fragile]{Usando as cores do tema}

  Se desejar usar uma das cores do tema para realçar parte do texto, pode-se usar o comando {\small\verb|\usebeamercolor|}.
  \medskip

  Por exemplo, para escrever a palavra {\usebeamercolor[fg]{block title}teste} na cor ``\texttt{block title fg}'', usamos o comando

  \begin{center}
    {\small\verb|{\usebeamercolor[fg]{block title}teste}|}
  \end{center}

  Para facilitar, podemos definir uma macro para isso, como fizemos com a cor ``\texttt{block title example fg}'':
 
  \begin{center}
    {\footnotesize\verb|\newcommand{\highlight}[1]{{\usebeamercolor[fg]{block title example}#1}}|}
  \end{center}

  Assim, para escrever \highlight{teste} nesta cor, basta digitar

  \begin{center}
    {\small\verb|\highlight{teste}|}
  \end{center}
\end{frame}

\begin{frame}[fragile]{Barra de progresso}

  Observe a barra de progresso acima.
  \pause
  
  \bigskip
  Não muda quando usamos {\verb|\pause|} ou \emph{overlays} em um mesmo \emph{frame}.

\end{frame}

\begin{frame}{Mais sobre a barra de progresso}

  Mas, naturalmente, a barra de progresso muda quando mudamos de \emph{frame}.
  
  \bigskip
  Quando o número total de \emph{frames} muda, é necessário compilar o arquivo latex duas vezes, para que esse número total seja devidamente atualizado e, com isso, a barra de progresso, também.

\end{frame}

\begin{frame}{Títulos muito longos são quebrados automaticamente} 
  
  No caso do título de um \emph{frame} ser muito comprido, o tema quebra as linhas automaticamente.
  \medskip

  O limite de caracteres depende da opção da posição da logomarca do IM, de estar em topo ou no pé de página.
  \medskip

  Mais detalhes a seguir.

\end{frame}

\begin{frame}{1\-2\-3\-4\-5\-6\-7\-8\-9\-1\-1\-2\-3\-4\-5\-6\-7\-8\-9\-2\-1\-2\-3\-4\-5\-6\-7\-8\-9\-3\-1\-2\-3\-4\-5\-6\-7\-8\-9\-4\-1\-2}

  Veja que, com a logomarca do IM em cima, cada linha do título do \emph{frame} abriga, no máximo, 42 caracteres, o que é 16\% a menos do que os 50 caracteres exibidos na \href{run:apresentacao2.pdf}{Apresentação 2}.

\end{frame}

\begin{frame}{1\-2\-3\-4\-5\-6\-7\-8\-9\-1\-1\-2\-3\-4\-5\-6\-7\-8\-9\-2\-1\-2\-3\-4\-5\-6\-7\-8\-9\-3\-1\-2\-3\-4\-5\-6\-7\-8\-9\-4\-1\-2\-3\-4\-5\-6\-7\-8\-9\-5\-1\-2\-3\-4\-5\-6\-7\-8\-9\-6\-1\-2\-3\-4\-5\-6\-7\-8\-9\-7\-1\-2\-3\-4\-5\-6\-7\-8\-9\-8\-1\-2\-3\-4\-5\-6\-7\-8\-9\-9\-1\-2\-3\-4\-5\-6\-7\-8\-9}

  Caso o título do \emph{frame} ultrapasse o número de 42 caracteres, o título é quebrado automaticamente e todas as linhas ficam limitadas a este mesmo número de caracteres.

\end{frame}

\begin{frame}

  \centerline{\Large Fim!}

\end{frame}

\end{document}